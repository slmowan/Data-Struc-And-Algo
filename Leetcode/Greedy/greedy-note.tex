\documentclass{article}
\usepackage{graphicx}
\usepackage{amsmath} % For math formatting
\usepackage{amssymb}

\begin{document}

\title{Greedy Algorithm}
\author{Wan}
\date{\today}
\maketitle

\noindent

\section{Introduction}

\section{The optimal substructure}

\section{Greedy Choice Property}



\section 
Idea: select local optimum at each step to get global optimum (Greedy Choice Property).

A special case of DP.

Goal: to find the best solution among a set of possible solutions, generally used in the optimization problem.

Verify the correctness of Greddy: Induction or proof by contradiction.

%Greedy choice不能反悔,一旦做出一个贪心选择,就不再回头重新考虑之前的选择。这是与其他算法(例如动态规划或回溯)的主要区别,其中可能需要重新考虑之前的决策。

% 要弄清楚参数之间的关系,比如在活动选择问题中,两个参数的关系是:0 <= s_i < f_i < inf

% 需要证明的是贪心算法是否能对所有的输入实例有正确的解。

% 证明贪心算法不正确:给出一个反例,即一个输入实例,使得贪心算法不能得到正确的解。

\section{Correctness Proof}

\section{Examples}

\subsection*{Activity Selection Problem}
\subsection*{Hauffman Coding}

\end{document}